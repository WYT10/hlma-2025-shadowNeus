\documentclass[12pt]{article}

% Essential packages for encoding, math, and formatting
\usepackage[utf8]{inputenc}
\usepackage[T1]{fontenc}
\usepackage{amsmath, amssymb, amsthm, mathtools} % Enhanced math support
\usepackage{geometry}
\geometry{a4paper, margin=1in}
\usepackage{hyperref}
\hypersetup{
    colorlinks=true,
    linkcolor=blue,
    citecolor=blue,
    urlcolor=blue
}
\usepackage{microtype} % Improved typography, prevents hyphenation issues
\usepackage{tocloft} % Table of contents customization
\usepackage{abstract} % For abstract environment
\usepackage{setspace} % For line spacing control
\usepackage{mathptmx} % Times-compatible math font (loaded last)

% Prevent hyphenation for specific words
\hyphenation{ShadowNeuS}

% Customize table of contents
\renewcommand{\cftsecleader}{\cftdotfill{\cftdotsep}}
\setlength{\cftbeforesecskip}{0.5em}

% Define theorem environments for mathematical content
\newtheorem{theorem}{Theorem}[section]
\newtheorem{lemma}{Lemma}[section]
\newtheorem{corollary}{Corollary}[section]

\begin{document}

% Title page
\begin{titlepage}
    \centering
    \vspace*{1cm}
    {\huge \textbf{From Projection to Perception: A Mathematical Exploration of Shadow-based Neural Reconstruction}\par}
    \vspace{1.5cm}
    {\normalsize
    A research report submitted to the Scientific Committee of the Hang Lung Mathematics Award\par}
    \vspace{1cm}
    {\normalsize \textbf{Team Number}\par 2596873\par}
    \vspace{0.5cm}
    {\normalsize \textbf{Team Members}\par Wong Yuk To, Hung Kwong Lam, Cheung Tsz Lung, Chan Ngo Tin, Zhou Lam Ho\par}
    \vspace{0.5cm}
    {\normalsize \textbf{Teacher}\par Mr. Chan Ping Ho\par}
    \vspace{0.5cm}
    {\normalsize \textbf{School}\par Po Leung Kuk Celine Ho Yam Tong College\par}
    \vspace{0.5cm}
    {\normalsize \textbf{Date}\par \today\par}
    \vspace{2cm}

% Abstract
\begin{abstract}
\raggedright % Prevent hyphenation in abstract
This paper explores ShadowNeuS [LWX23], a neural network that reconstructs 3D geometry from single-view camera images using shadow and light cues. Unlike traditional 3D reconstruction methods relying on multi-view cameras or sensors, ShadowNeuS leverages a neural signed distance field (SDF) for accurate 3D geometry reconstruction. Analysis of the training process reveals deep connections to projective geometry, spatial reasoning in $\mathbb{R}^3$, and the network's perception of three-dimensional space.
\end{abstract}

\end{titlepage}
% Table of contents
\tableofcontents
\newpage

% Main content
\section{Introduction}
% Placeholder for introduction content
% Example mathematical content:
% \begin{theorem}
% Let \( f: \mathbb{R}^3 \to \mathbb{R} \) be a signed distance function defined for a surface \( S \). Then, the gradient \( \nabla f \) satisfies...
% \end{theorem}

% References (retaining original tabular method)
\section*{References}
\begin{tabular}{@{}p{0.1\textwidth} p{0.9\textwidth}}
{[LWX23]} & Jingwang Ling, Zhibo Wang, Feng Xu. \textit{ShadowNeuS: Neural SDF Reconstruction by Shadow Ray Supervision}. arXiv: \href{https://arxiv.org/abs/2211.14086}{2211.14086}, 2023.
\end{tabular}

\end{document}