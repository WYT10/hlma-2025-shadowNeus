\documentclass[12pt]{article}

% Essential packages for encoding, math, and formatting
\usepackage[utf8]{inputenc}
\usepackage[T1]{fontenc}
\usepackage{amsmath, amssymb, amsthm, mathtools, bm} % Enhanced math support
\usepackage{geometry}
\geometry{a4paper, margin=0.8in}
\usepackage{hyperref}
\hypersetup{
    colorlinks=true,
    linkcolor=blue,
    citecolor=blue,
    urlcolor=blue
}
\usepackage{microtype} % Improved typography
\usepackage{tocloft} % Table of contents customization
\usepackage{abstract} % For abstract environment
\usepackage{mathptmx} % Times-compatible math font
\usepackage{xcolor} % For colored text
\usepackage{tikz} % For mathematical diagrams
\usepackage{enumitem} % Better list formatting

% Custom math commands for enhanced notation
\newcommand{\R}{\mathbb{R}}
\newcommand{\C}{\mathbb{C}}
\newcommand{\N}{\mathbb{N}}
\newcommand{\Z}{\mathbb{Z}}
\newcommand{\vect}[1]{\bm{#1}}
\newcommand{\transpose}[1]{#1^{\mathsf{T}}}
\newcommand{\norm}[1]{\left\|\,#1\,\right\|}
\newcommand{\abs}[1]{\left|\,#1\,\right|}
\newcommand{\inner}[2]{\langle\,#1,\,#2\,\rangle}

% Define theorem environments
\theoremstyle{definition}
\newtheorem{definition}{Definition}[section]
\newtheorem{theorem}{Theorem}[section]
\newtheorem{lemma}{Lemma}[section]
\newtheorem{proposition}{Proposition}[section]
\newtheorem{remark}{Remark}[section]

\begin{document}

% Title page
\begin{titlepage}
    \centering
    \vspace*{1cm}
    {\huge \textbf{From Projection to Perception: \\ A Mathematical Exploration of \\ Shadow-based Neural Reconstruction}\par}
    \vspace{1.5cm}
    {\normalsize
    A research report submitted to the Scientific Committee of the Hang Lung Mathematics Award\par}
    \vspace{1cm}
    {\normalsize \textbf{Team Number}\par 2596873\par}
    \vspace{0.5cm}
    {\normalsize \textbf{Team Members}\par Wong Yuk To, Hung Kwong Lam \\ Cheung Tsz Lung, Chan Ngo Tin, Zhou Lam Ho\par}
    \vspace{0.5cm}
    {\normalsize \textbf{Teacher}\par Mr. Chan Ping Ho\par}
    \vspace{0.5cm}
    {\normalsize \textbf{School}\par Po Leung Kuk Celine Ho Yam Tong College\par}
    \vspace{0.5cm}
    {\normalsize \textbf{Date}\par \today\par}
    \vspace{2cm}

% Abstract
\begin{abstract}
\raggedright
This paper explores \textsc{ShadowNeuS} \hyperlink{[LWX23]}{[LWX23]}, a neural network that reconstructs 3D geometry from single-view camera images using shadow and light cues. Unlike traditional 3D reconstruction methods relying on multi-view cameras or sensors, \textsc{ShadowNeuS} leverages a neural signed distance field (SDF) for accurate 3D geometry reconstruction. We analyze the training process and uncover its connections to projective geometry, spatial reasoning in $\R^3$, and the neural network's learned geometric representation of space.
\end{abstract}

\end{titlepage}

% Table of contents
\tableofcontents

\newpage
\section{Background}

\subsection{What is 3D Reconstruction from Images?}

The goal of 3D reconstruction is to recover the structure of a 3D scene using only 2D images.

\begin{definition}[3D Scene Representation] ~\\
A 3D scene is represented by a set of points $\vect{P} = [P_x, P_y, P_z]^\mathsf{T} \in \R^3$ in Euclidean space.
\end{definition}

\begin{definition}[Image Projection] ~\\
Each image $I_n$ of the 3D scene records a set of pixel coordinates $\vect{p} = [p_x, p_y]^\mathsf{T} \in \R^2$.
\end{definition}

The process of capturing a 3D point in a 2D image $I_n$ can be modeled as a projection function $\pi_n$:

\vspace{0.5em}
\begin{equation}
\boxed{\pi_n: \R^3 \to \R^2, \quad [P_x, P_y, P_z]^\mathsf{T} \mapsto [p_x, p_y]^\mathsf{T}} \label{eq:proj}
\end{equation}
\vspace{0.5em}

This projection function represents how a camera maps a 3D point to a 2D pixel in the $n$-th image.

To reconstruct the 3D scene, we need to solve the \textbf{inverse problem} $\pi_n^{-1}$:

\vspace{0.5em}
\begin{equation}
\boxed{\pi_n^{-1}(\vect{p}) = \left\{ \vect{P} \in \R^3 \mid \pi_n(\vect{P}) = \vect{p} \right\}} \label{eq:invproj}
\end{equation}
\vspace{0.5em}

However, this inverse problem is typically \textbf{ill-posed}, as multiple 3D points may project to the same 2D pixel, leading to ambiguity. We will detail this in Section \ref{sec:inverse}.

\subsection{Information Encoded in 2D Images}

A 2D image $I_n$ can provide multiple types of information encoded as mathematical structures:

\subsubsection*{Information Available from an Image:}
\begin{enumerate}[label=(\roman*)]
    \item \textbf{Pixel coordinates}: $\vect{p} = [p_x, p_y]^\mathsf{T} \in \R^2$, represents the spatial location of each pixel
    
    \item \textbf{Color values}: $C_n(\vect{p}) = [r, g, b]^\mathsf{T} \in [0,1]^3$, represents the RGB tristimulus values
    
    \item \textbf{RGB gradient matrix}:
    \begin{equation}
        \nabla C_n(\vect{p}) = \begin{bmatrix}
            \frac{\partial r}{\partial p_x} & \frac{\partial r}{\partial p_y} \\[0.5em]
            \frac{\partial g}{\partial p_x} & \frac{\partial g}{\partial p_y} \\[0.5em]
            \frac{\partial b}{\partial p_x} & \frac{\partial b}{\partial p_y}
        \end{bmatrix} \in \R^{3 \times 2} \label{eq:gradient}
    \end{equation}
    This Jacobian matrix captures local intensity variations, indicating edges or texture information.
    
    \item \textbf{Learned feature embedding}: $\phi(I_n)(\vect{p}) \in \R^d$, represents high-dimensional features extracted via neural networks like CNNs
\end{enumerate}

These data structures result from projecting 3D geometry through camera optics, where $C_n(\vect{p})$ corresponds to visible surface reflectance and $\nabla C_n(\vect{p})$ encodes geometric boundaries.

\newpage    

\subsection{The Forward Projection: From 3D World to 2D Image}

We formalize the perspective projection process using homogeneous coordinates and transformation matrices.

\subsubsection*{Camera Parameter Matrices:}

\begin{definition}[Extrinsic Parameters] ~\\
The world-to-camera transformation is characterized by:
\begin{align}
\text{Camera center: } &\quad \vect{C} = [C_x, C_y, C_z]^\mathsf{T} \in \R^3 \\
\text{Rotation matrix: } &\quad R = \begin{bmatrix}
    r_{11} & r_{12} & r_{13} \\
    r_{21} & r_{22} & r_{23} \\
    r_{31} & r_{32} & r_{33}
\end{bmatrix} \in \text{SO}(3) \\
\text{Translation vector: } &\quad \vect{t} = -R \vect{C} \in \R^3
\end{align}
\end{definition}

\begin{definition}[Intrinsic Parameters] ~\\
The camera's internal geometry is encoded by:
\begin{equation}
K = \begin{bmatrix}
    f_x & 0 & c_x \\
    0 & f_y & c_y \\
    0 & 0 & 1
\end{bmatrix} \in \R^{3 \times 3} \label{eq:intrinsic}
\end{equation}
where $(f_x, f_y)$ are focal lengths in pixels and $(c_x, c_y)$ is the principal point.
\end{definition}

\subsubsection*{Forward Projection Pipeline:}

\begin{proposition}[Perspective Projection Transform] ~\\
The complete forward projection involves three sequential transformations:
\begin{enumerate}[label=\textbf{Step \arabic*:}]
    \item \textbf{World to camera coordinates}
    \begin{equation}
        \vect{P}_{\text{cam}} = R \vect{P} + \vect{t}
    \end{equation}
    
    \item \textbf{Camera to image coordinates}
    \begin{equation}
        \vect{P}_{\text{hom}} = K \vect{P}_{\text{cam}} = \begin{bmatrix} p_x' \\ p_y' \\ z' \end{bmatrix}
    \end{equation}
    
    \item \textbf{Perspective division}
    \begin{equation}
        \vect{p} = \begin{bmatrix} p_x \\ p_y \end{bmatrix} = \frac{1}{z'} \begin{bmatrix} p_x' \\ p_y' \end{bmatrix}, \quad z' \neq 0
    \end{equation}
\end{enumerate}
\end{proposition}

The complete transformation matrix can be expressed as:

\vspace{0.5em}
\begin{equation}
\boxed{\vect{P}_{\text{hom}} = K [R \mid \vect{t}] \begin{bmatrix} \vect{P} \\ 1 \end{bmatrix} , \quad \vect{p} = \frac{1}{z'} \begin{bmatrix} p_x' \\ p_y' \end{bmatrix}} \label{eq:forward}
\end{equation}
\vspace{0.5em}

\newpage

\subsection{The Inverse Problem: From 2D Image to 3D World} \label{sec:inverse}

We now tackle the fundamental challenge of inverting the projection function.

\begin{lemma}[Camera Ray Parametrization] ~\\
Given a pixel $\vect{p} = [p_x, p_y]^\mathsf{T}$ and camera parameters $(K, R, \vect{C})$, the corresponding 3D points form a ray:

\vspace{0.25em}
\begin{equation}
\boxed{\vect{P}(\lambda) = \vect{C} + \lambda \cdot \vect{d}, \quad \lambda > 0} \label{eq:camera_ray}
\end{equation}
\vspace{0.25em}

where the ray direction is:

\vspace{0.25em}
\begin{equation}
\boxed{\vect{d} = R^{-1} K^{-1} \begin{bmatrix} p_x \\ p_y \\ 1 \end{bmatrix}} \label{eq:ray_direction}
\end{equation}
\vspace{0.25em}
\begin{remark}[Normalization] ~\\
The direction vector $\vect{d}$ can optionally be normalized to unit length for physical ray tracing but not strictly necessary for the ray parametrization.
\end{remark}
\end{lemma}

\begin{proof}
Starting from the forward projection equation \eqref{eq:forward}:
\begin{align}
K (R \vect{P} + \vect{t}) &= z' \begin{bmatrix} p_x \\ p_y \\ 1 \end{bmatrix} \\
R \vect{P} + \vect{t} &= z' K^{-1} \begin{bmatrix} p_x \\ p_y \\ 1 \end{bmatrix} \\
\vect{P} &= R^{-1} \left( z' K^{-1} \begin{bmatrix} p_x \\ p_y \\ 1 \end{bmatrix} - \vect{t} \right)
\end{align}

Since $\vect{t} = -R \vect{C}$, we have $-R^{-1} \vect{t} = \vect{C}$. Setting $\lambda = z'$:
\begin{equation}
\vect{P}(\lambda) = \vect{C} + \lambda \cdot R^{-1} K^{-1} \begin{bmatrix} p_x \\ p_y \\ 1 \end{bmatrix}
\end{equation}
\end{proof}

\begin{proposition}[Ill-posed Nature of Single-View Reconstruction] ~\\
The inverse projection problem is fundamentally \textbf{ill-posed} because:
\begin{enumerate}[label=(\alph*)]
    \item The depth parameter $\lambda$ is undetermined
    \item Each pixel $\vect{p}$ defines a ray of infinitely many possible 3D points
    \item Additional constraints are required for unique reconstruction
\end{enumerate}
\end{proposition}

\newpage

\subsection{Cues for Solving the Inverse Problem}

To achieve unique reconstruction, we require additional information such as:
\begin{itemize}
    \item \textbf{Stereo correspondence}: Multiple viewpoints providing triangulation
    \item \textbf{Depth sensors}: Direct measurement of $\lambda$
    \item \textbf{Shadow constraints}: Geometric relationships via light ray intersections
\end{itemize}

\section{Shadows as a Geometric Constraint}

We now introduce how shadows, often considered a nuisance in image understanding, can instead be leveraged as powerful geometric constraints. By formalizing light transport and occlusion, we derive conditions that allow recovery of 3D structure from single images.

\subsection{Light Ray and Shadow Geometry}

\begin{definition}[Light Ray] ~\\
Given a light source $\vect{L} \in \mathbb{R}^3$ and a point $\vect{P} \in \mathbb{R}^3$, the light ray from $\vect{L}$ to $\vect{P}$ is the segment:
\begin{equation}
\boxed{r(t) = \vect{L} + t(\vect{P} - \vect{L}), \quad t \in [0,1]} \label{eq:light_ray}
\end{equation}
\end{definition}

\begin{definition}[Shadow Occlusion] ~\\
A point $\vect{P}$ is in shadow if there exists some $t \in (0,1)$ such that $r(t)$ intersects a surface $\mathcal{S}$:
\begin{equation}
\boxed{\exists t \in (0,1): r(t) \cap \mathcal{S} \neq \emptyset}
\end{equation}
\end{definition}

\begin{remark}[Physical Interpretation] ~\\
The open interval $(0,1)$ corresponds to obstructions between the light source and the point. Intersection in this interval implies occlusion and shadowing.
\end{remark}

\subsection{Shadow Boundary and Surface Partitioning}

\begin{theorem}[Tangency Condition] ~\\
A point $\vect{Q} \in \mathcal{S}$ lies on the shadow boundary if and only if the light ray is tangent to the surface at that point:
\begin{equation}
\boxed{(\vect{Q} - \vect{L}) \cdot \vect{n}(\vect{Q}) = 0} \label{eq:tangency}
\end{equation}
where $\vect{n}(\vect{Q})$ is the unit surface normal at $\vect{Q}$.
\end{theorem}

\begin{remark}
The dot product condition expresses that the vector from the light source to the point is orthogonal to the surface normal—indicating grazing incidence.
\end{remark}

\begin{proposition}[Shadow Boundary Set] ~\\
The 3D shadow boundary is defined as:
\begin{equation}
\boxed{\mathcal{B} = \{\vect{Q} \in \mathcal{S} \mid (\vect{Q} - \vect{L}) \cdot \vect{n}(\vect{Q}) = 0\}} \label{eq:boundary_set}
\end{equation}
\end{proposition}

\begin{proposition}[Surface Illumination Partition] ~\\
The surface $\mathcal{S}$ is partitioned into:
\begin{align}
\mathcal{S}_{\text{lit}} &= \left\{ \vect{P} \in \mathcal{S} \mid (\vect{P} - \vect{L}) \cdot \vect{n}(\vect{P}) > 0 \right\} && \text{(illuminated)} \\
\mathcal{A} &= \left\{ \vect{P} \in \mathcal{S} \mid (\vect{P} - \vect{L}) \cdot \vect{n}(\vect{P}) < 0 \right\} && \text{(attached shadow)} \\
\mathcal{B} &= \left\{ \vect{P} \in \mathcal{S} \mid (\vect{P} - \vect{L}) \cdot \vect{n}(\vect{P}) = 0 \right\} && \text{(shadow boundary)}
\end{align}
\end{proposition}

\begin{remark}
This partition reflects the angular relationship between the surface normal and the light direction, encoding geometric visibility information.
\end{remark}

\newpage

\subsection{Cast Shadows on Secondary Surfaces}

\begin{definition}[Cast Shadow Region] ~\\
Given an occluding surface $\mathcal{S}_1$ and a receiving surface $\mathcal{S}_2$, the cast shadow region is:
\begin{equation}
\boxed{
\mathcal{C}_{1 \to 2} = \left\{ \vect{P} \in \mathcal{S}_2 \mid \exists t \in (0,1) \text{ such that } \vect{L} + t(\vect{P} - \vect{L}) \in \mathcal{S}_1
\right\}
}
\end{equation}
\end{definition}

\begin{definition}[Cast Shadow Boundary] ~\\
The boundary of the cast shadow on $\mathcal{S}_2$ is given by:
\begin{equation}
\boxed{
\partial\mathcal{C}_{1 \to 2} = \left\{ \vect{P} = \vect{Q} + s(\vect{Q} - \vect{L}) \mid \vect{Q} \in \mathcal{B}_1, s > 0 \right\}
}
\end{equation}
where $\mathcal{B}_1$ is the shadow boundary on the occluding surface $\mathcal{S}_1$.
\end{definition}

\begin{remark}
Cast shadow boundaries are formed by rays tangent to the occluder, extending toward the receiver.
\end{remark}

\subsection{Shadows as Cues for 3D Reconstruction}

Shadows, especially their boundaries, encode constraints that can be leveraged to resolve ambiguities in monocular depth estimation.

\begin{proposition}[Geometric Information Encoded in Shadows] ~\\
Shadows provide four types of geometric cues:
\begin{enumerate}[label=(\roman*)]
\item \textbf{Surface orientation}: Attached shadows satisfy $(\vect{P} - \vect{L}) \cdot \vect{n}(\vect{P}) < 0$
\item \textbf{Boundary tangency}: Shadow boundaries satisfy $(\vect{Q} - \vect{L}) \cdot \vect{n}(\vect{Q}) = 0$
\item \textbf{Relative depth}: Cast shadows reveal spatial relationships between objects
\item \textbf{Occluded structure}: Shadowed areas imply presence of obstructing geometry
\end{enumerate}
\end{proposition}

\begin{theorem}[Single-View Depth Recovery via Shadow Constraints] ~\\
Let a pixel $\vect{p}$ on the image lie on the projected shadow boundary. Its corresponding 3D point must lie on the camera ray:
\begin{equation}
\vect{P} = \vect{C} + \lambda \vect{d}
\end{equation}
Imposing the tangency condition from equation~\eqref{eq:tangency} gives:
\begin{equation}
\boxed{(\vect{C} + \lambda \vect{d} - \vect{L}) \cdot \vect{n}(\vect{C} + \lambda \vect{d}) = 0} \label{eq:shadow_constraint}
\end{equation}
\end{theorem}

This equation expresses that the unknown 3D point lies on both the shadow boundary and the camera ray, enforcing a tangency condition that geometrically constrains its depth. By solving this nonlinear equation, we can obtain a unique value $\lambda^*$ that determines the 3D point $\vect{P}^*$.

\begin{proposition}[Cast Shadow Consistency Check] ~\\
To validate the reconstruction $\vect{P}^*$, extend the light ray from $\vect{P}^*$ and check whether its shadow projection matches the observed image:
\begin{equation}
\boxed{
\pi(\vect{P}^* + s(\vect{P}^* - \vect{L})) \in \Omega_{\text{shadow}}^{\text{obs}}, \quad s > 0
}
\end{equation}
where $\Omega_{\text{shadow}}^{\text{obs}} \subset \R^2$ denotes the observed shadow region in the image.
\end{proposition}

\begin{remark}[Geometric Paradigm Shift] ~\\
This framework reinterprets shadows not as photometric noise, but as reliable geometric constraints—enabling single-view 3D reconstruction without requiring texture, stereo pairs, or depth sensors.
\end{remark}

\newpage

\subsection{Limitations of Shadow-Based Reconstruction}

Shadows provide valuable geometric constraints but face challenges in certain scenarios. We suggest some key limitations below.
\begin{enumerate}[label=\arabic*.]
    \item \textbf{Diffuse Lighting}: Diffuse or ambient light eliminates distinct shadow boundaries. The tangency condition $(\vect{Q} - \vect{L}) \cdot \vect{n}(\vect{Q}) = 0$ (equation~\eqref{eq:tangency}) requires a point light source $\vect{L}$. Without it, the boundary set $\mathcal{B}$ is undefined, making equation~\eqref{eq:shadow_constraint} unsolvable.
    
    \item \textbf{Self-Shadowing}: In complex scenes, multiple occluders $\mathcal{S}_1, \mathcal{S}_2$ create overlapping shadows. For a point $\vect{P} \in \mathcal{S}_2$, the light ray $r(t) = \vect{L} + t(\vect{P} - \vect{L})$ may have multiple intersections $t_1, t_2 \in (0,1)$, leading to ambiguous cast shadow regions $\mathcal{C}_{1 \to 2}$ and under-constrained depth $\lambda$ in equation~\eqref{eq:shadow_constraint}.
    
    \item \textbf{Unknown Light Position}: Accurate $\vect{L}$ is critical. An error $\Delta \vect{L}$ shifts the tangency condition to:
    \[
    (\vect{Q} - (\vect{L} + \Delta \vect{L})) \cdot \vect{n}(\vect{Q}) = 0,
    \]
    yielding incorrect $\mathcal{B}$ and erroneous $\vect{P}^* = \vect{C} + \lambda^* \vect{d}$. Estimating $\vect{L}$ from a single image is ill-posed.
    
    \item \textbf{Non-Lambertian Surfaces}: Specular surfaces distort shadow boundaries, misaligning observed $C_n(\vect{p})$ with geometric boundaries. Errors in $\nabla C_n(\vect{p})$ (equation~\eqref{eq:gradient}) invalidate the shadow constraint:
    \[
    (\vect{C} + \lambda \vect{d} - \vect{L}) \cdot \vect{n}(\vect{C} + \lambda \vect{d}) = 0.
    \]
    
    \item \textbf{Computational Complexity}: Solving equation~\eqref{eq:shadow_constraint} requires minimizing:
        \[\abs{(\vect{C} + \lambda \vect{d} - \vect{L}) \cdot \vect{n}(\vect{C} + \lambda \vect{d})}\]
        which is computationally intensive. Noisy shadow edges lead to local minima, producing incorrect $\vect{P}^*$.
\end{enumerate}

\begin{remark}
Shadow-based reconstruction remains challenging due to the above limitations. However, exploring and extracting information encoded by shadows is highly useful for 3D reconstruction.
\end{remark}

\section{}













\newpage

\section*{References}
\begin{tabular}{@{}p{0.1\textwidth} p{0.9\textwidth}}
\hypertarget{[LWX23]}{[LWX23]} & Jingwang Ling, Zhibo Wang, Feng Xu. \textit{ShadowNeuS: Neural SDF Reconstruction by Shadow Ray Supervision}. arXiv: \href{https://arxiv.org/abs/2211.14086}{2211.14086}, 2023.
\end{tabular}

\end{document}